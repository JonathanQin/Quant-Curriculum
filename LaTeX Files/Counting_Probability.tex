\documentclass{article}
\usepackage{graphicx} % Required for inserting images
\usepackage{blindtext}
\usepackage[a4paper, total={6in, 9.5in}]{geometry}
\usepackage{enumitem}


\title{

\begin{center} \textbf{Quant Curriculum Week 1}

Counting and Probability Fundamentals \end{center}
}

\author{Jonathan Qin}
\date{February 23, 2023}

\begin{document}

\maketitle

\section{Introduction to Curriculum}

Often during the recruiting process, you will be tested with a mysterious technical interview that you aren't sure how to prepare for (especially for quant related positions). However, a universal skill required for quant trader, researcher, and dev interviews alike is a strong mathematical background and problem-solving intuition. This is no surprise, as these qualities translate naturally into programming and trading. In fact, quant firms rarely expect any in depth finance knowledge, and put new hires through rigorous trading bootcamps regardless. The goal of the curriculum will be to introduce and build the mathematical knowledge and problem solving skills needed for any technical interview. 

\subsection{General Principles for Problem Solving during Interviews}
\begin{itemize}
    \item Listen carefully, look for clues 
    \item Verbalize and rationalize 
    \item Make reasonable assumptions
\end{itemize}

\paragraph{Problem 1.1} Dropping Eggs
\newline
\newline
A building has 100 floors. One of the floors is the highest floor an egg can be dropped from without breaking.
If an egg is dropped from above that floor, it will break. If it is dropped from that floor or below, it will be completely undamaged, and you can drop the egg again. Given two eggs, find the highest floor where an egg can be dropped without breaking with as few drops as possible. 
\newline
\subsection{Old Mission Capital Problems}
\begin{enumerate}
    \item If you repeatedly flip a single coin, what is the expected number of flips it will take you to flip heads twice consecutively?
    \item Four friends are going to play a game together - Alice and Bob are team #1, Charlie and Denise are team #2. They draw a perfect, large circle in a field, and each of the 4 will randomly and independently choose a spot to stand on the circle. Each team will pick up a rope, pull it tight between them, connecting the two players. What is the probability the two teams' ropes will cross over each other or intersect?
    \item (Unrelated to CP) Bob and Carl play rock paper scissors, where each player simultaneously chooses Rock, Paper, or Scissors. A player wins, loses, or ties based on conventional rules. Given that they play 10 times and you know the following:
    \begin{enumerate}
        \item Bob used rock 3 times, scissors 6 times, and paper 1 time
        \item Carl used rock 2 times, scissors 4 times, and paper 4 times
        \item there were no ties in all 10 games
    \end{enumerate}
    Who wins and by how many games?
    
\end{enumerate}



\section{Counting Strategies}

Often in problem-solving, there will be a need to count possible outcomes and compute their probabilities. Being able to compute fast (as a human) is critical in making fast decisions, creative problem solving, and writing sleek code. Therefore, we will first cover some elegant ways to compute the size of sets of interest.

\paragraph{Problem 2.1} Counting Numbers (2006 AMC 10A Problem 21)
\newline
\newline
How many four-digit positive integers have at least one digit that is a $2$ or a $3$?
\newline
\newline
\paragraph{Problem 2.2} Triangles (2017 AMC 10A Problem 23)
\newline\newline
How many triangles with positive area have all their vertices at points $(i,j)$ in the coordinate plane, where $i$ and $j$ are integers between $1$ and $5$, inclusive?
\newline
\newline
\subsection{Three Fundamental Counting Techniques}
\newline
There are three fundamental counting techniques: constructive counting, casework, and complementary counting.
\newline \begin{enumerate}

\item{\textbf{Constructive Counting:} a counting technique that involves constructing an item belonging to a set. Along with the construction, one counts the total possibilities of each step and assembles these to enumerate the full set.}

\item{\textbf{Counting by Casework:} a counting method that involves splitting a problem into several parts, counting these parts individually, then adding together the totals of each part.}

\item{\textbf{Complementary Counting:} a counting strategy for counting where you count what you don't want and subtract that from the total number of possibilities to arrive at the answer. More formally, if $B$ is a subset of $A$, complementary counting exploits the property that $|B| = |A| - |B^c|$, where $B^c$ is the complement of $B$. \textbf{Complementary probability} is the probability equivalent.}

\end{enumerate}
\newline
\newline
\paragraph{Problem 2.3} License Plates (2002 AIME I Problem 1)
\newline
\newline
Many states use a sequence of three letters followed by a sequence of three digits as their standard license-plate pattern. Given that each three-letter three-digit arrangement is equally likely, the probability that such a license plate will contain at least one palindrome (a three-letter arrangement or a three-digit arrangement that reads the same left-to-right as it does right-to-left) is $\dfrac{m}{n}$, where $m$ and $n$ are relatively prime positive integers. Find $m+n.$
\newline
\newline
\newline
\subsection{Common Counting Strategies}
\begin{enumerate}
\item{\textbf{Principle of Inclusion Exclusion:} an organized method/formula to find the number of elements in the union of a given group of sets, the size of each set, and the size of all possible intersections among the sets.

\item{\textbf{Pigeonhole Principle:} if $n+1$ or more pigeons are placed into $n$ holes, one hole must contain two or more pigeons.

\end{enumerate}


\paragraph{Problem 2.4} Handshakes 
\newline
\newline
There are $n$ guests at party. Guests shake hands when they meet other guests. Some might be more social and shake hands with a lot of other guests, while others may be less social and shake hands with few or no other guests. Should there be at least two guests with the same number of handshakes at the party?
\newline
\newline
\newline
\paragraph{Problem 2.5 (Hard)} Sitting Order (2020 AIME II Problem 9)
\newline
\newline
While watching a show, Ayako, Billy, Carlos, Dahlia, Ehuang, and Frank sat in that order in a row of six chairs. During the break, they went to the kitchen for a snack. When they came back, they sat on those six chairs in such a way that if two of them sat next to each other before the break, then they did not sit next to each other after the break. Find the number of possible seating orders they could have chosen after the break.
\newline
\newline
\newline
\subsection{More Counting Strategies}
\begin{enumerate}
\item{\textbf{Correspondence:} a relation between two sets such that each member in one set corresponds to $n$ members in the other set, where $n$ commonly equals $1$

\item{\textbf{Generating Functions:} creating a power series whose coefficients, $c_0, c_1, c_2, \ldots$, give the terms of a sequence which is of interest. Therefore the power series (i.e. the generating function) is $c_0 + c_1 x + c_2 x^2 + \cdots$ and the sequence is $c_0, c_1, c_2,\ldots$

\item{\textbf{Recursion:} defining something (usually a sequence or function) in terms of previously defined values
\newline
\newline
\end{enumerate}
\paragraph{Problem 2.6} Distributing Cookies
\begin{enumerate} [label=\alph*)]
    \item In how many different ways can eight identical cookies be distributed among three distinct children?
    \item How many ways can the candy be distributed if each child receives at least two but no more than four cookies?
\end{enumerate}
\newline
\newline
\newline
\paragraph{Problem 2.7} Meme Stock
\begin{enumerate} [label=\alph*)]
    \item Jonathan's owns a meme-stock with behavior modeled by the function $f(t)$, which denotes the price at time $t = 0, 1, \ldots , n$
    \begin{equation}
        \(f(t) = 2f(t-1) + 5f(t-2)\)
    \end{equation}
    Given that the stock IPO was at \(f(0) = 1\) and rose to price \(f(1) = 8\), can you write a program or compute by hand \(f(10)\)?
    \item Is there a program that calculates \(f(t)\) for any \(t\) in \(O(1)\)? 
    \item What about for any \(f(t)\) in the form 
    \begin{equation}
        \(f(t) = c_1f(t-1) + c_2f(t-2)\)
    \end{equation}
    with known \(f(1)\) and \(f(0)\)?
\end{enumerate}
\newline
\newline
\newline

\section{Introduction to Probability Theory}

Computing probabilities of outcomes is another common problem. We define:
\begin{enumerate}
    \item Sample space $S$: set of all possible outcomes
    \item Event: particular set of outcomes in the sample space
    \item Mutually exclusive: $A \cap B = \emptyset$ and \(P(A\cup B) = P(A) + P(B)\)
    \item Independent: \(P(A)P(B) = P(AB)\)
    \item Conditional Probability: \(P(B|A) = \frac{P(AB)}{P(A)}\)
    \item Bayes Theorem: \(P(A|B)= \frac{P(A)P(B|A)}{P(B)}\)
    \item Expectation: \(E[x] = \sum_{0}^{i}{p_ix_i\)
    \item Variance: \(Var(x) = E[(x-\mu)^2] = E[x^2] - (E[x])^2\)
\end{enumerate}
\paragraph{Problem 3.1} Sleeping in Class
\newline
\newline
The probability Jonathan falls asleep during a two-hour class is \(\frac{3}{4}\). Jonathan is equally likely to fall asleep at any point in time, and once he falls asleep he will never wake up. What is the probability that Jonathan is asleep by the end of the first hour?
\newline
\newline
\paragraph{Problem 3.2} Caught Sleeping
\newline
\newline
Terry claims Jonathan fell asleep during class. Given that we know Jonathan has a \(\frac{1}{8}\) chance of falling asleep in class, and that Terry lies \(\frac{5}{6}\) of the time, what is the actual probability Jonathan fell asleep?
\newline
\newline
\paragraph{Problem 3.3} Basketball Scores 
\newline
\newline
A basketball player is taking 100 free throws. She scores one point if the ball passes through the hoop and 0 otherwise. She scored on her first throw and missed her second. For each following throw, her probability of making a point is the fraction of throws she has made so far. After 100 throws, what is the probability she has 50 points?
\newline
\newline
\paragraph{Problem 3.4} Coin Toss Game
\newline
\newline
Two players, A and B, alternatively toss a fair, with A going first. The game ends when a sequence of HT (head tails) appears, and the player who tosses the tails wins. What is the probability A wins the game?
\newline
\newline
\paragraph{Problem 3.4} Breaking Sticks
\newline
\newline
A stick is broken into 3 pieces, by randomly choosing two points along its unit length, and cutting it. What is the expected length of the middle part?
\newline
\newline
\subsection{Rules of Expectation}
\begin{enumerate}
    \item{\textbf{Random Variable:} functions that map each outcome in a sample space to a set of real numbers.}

    \item{\textbf{Linearity of Expectation:} property that the expected value of the sum of random variables is equal to the sum of their individual expected values, regardless of whether they are independent.}
\end{enumerate}

\newline
\newline

\paragraph{Problem 3.5} Breaking Sticks
\newline
\newline
There are 26 black(B) and 26 red(R) cards in a standard deck. A run is number of blocks of consecutive cards of the same color. For example, a sequence RRRRBBBRBRB of only 11 cards has 6 runs; namely, RRRR, BBB, R, B, R, B. Find the expected number of runs in a shuffled deck of cards.
\newline
\newline

\section {Summary}
Return to the Old Mission Capital Problems and see if you can solve them! Both problems can be solved by either counting or probability strategies. I will leave some additional problems to dwell over. Next curriculum will likely be more probability theory, statistics, and an introduction to Game Theory if we have time.

\section {Problem Set}
Here are some problems (not necessarily sorted by difficulty) related to this week's curriculum. Very optional, but also very good problems to test your understanding of the material
\newline
\begin{enumerate}
    \item A coin is tossed 10 times and the output written as a string. What is the expected number of HH? Note that in HHH, number of HH = 2. (eg: expected number of HH in 2 tosses is 0.25, 3 tosses is 0.5)
    \item p and q are two points chosen at random between 0 & 1. What is the probability that the ratio p/q lies between 1 & 2?
    \item An ant is standing on one corner of a cube & can only walk on the edges. The ant is drunk and from any corner, it moves randomly by choosing any edge! What is the expected number of edges the ant travels, to reach the opposite corner?
    \item You have 100 noodles in your soup bowl. You are told to take two ends of some noodles (each end on any noodle has the same probability of being chosen) in your bowl and connect them. You continue until there are no free ends. What is the expected number of loops? What is the probability of making one large loop which includes every noodle?
    \item You are taking out candies one by one from a jar that has 10 red candies, 20 blue candies, and 30 green candies in it. What is the probability that there are at least 1 blue candy and 1 green candy left in the jar when you have taken out all the red candies? (Candies of same color are indistinguishable!)
\end{enumerate}


\end{document}
