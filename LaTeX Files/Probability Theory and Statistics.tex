\documentclass{article}
\usepackage{graphicx} % Required for inserting images
\usepackage{blindtext}
\usepackage[a4paper, total={6in, 9.5in}]{geometry}
\usepackage{enumitem}


\title{

\begin{center} \textbf{Quant Curriculum Week 2}

Probability Theory and Statistics \end{center}
}

\author{Jonathan Qin}
\date{March 2, 2023}

\begin{document}

\maketitle

\section{Discrete and Continuous Distributions}

\subsection{Motivation}
We will begin by studying distribution functions widely used for quantitative modeling. Having a intuitive understanding of these functions and being able to discern the unique characteristics of a distribution is a valuable skill. 

\subsection{Definitions}
\begin{enumerate}
    \item Random Variable: can be discrete or continuous
    \item Cumulative distribution function: \(F(a) = P\{X \leq a\}\), \(\int_{-\infty}^{a} f(x) \,dx\)
    \item Probability mass function: \(p(x) = P\{X=x\}\) 
    \item Probability distribution function: \(f(x) = \frac{d}{dx}F(x)\)
    \item Expected value: \(E[X] = \sum_{x:p(x)>0} xp(x)\), \(\int_{-\infty}^{\infty} xf(x) \,dx\)
    \item Variance: \(var(X) = E[(X-E[X])^2] = E[X^2] - (E[X])^2\)
\end{enumerate}

\subsection{Discrete Random Variables}
\begin{enumerate}
    \item Uniform: \(P(x) = \frac{1}{b-a+1}\), \(E[X] = \frac{b+a}{2}\), \(var(X) = \frac{(b-a+1)^2 -1}{12}\)
    \item Binomial: 
    \item Poisson:
    \item Geometric:
    \item Negative Binomial:
\end{enumerate}

\subsection{Continuous Random Variables}
\begin{enumerate}
    \item 
\end{enumerate}

\end{document}